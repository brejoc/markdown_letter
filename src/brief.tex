\documentclass[%
  fontsize=12pt, % Schriftgröße
  version=last%  % Neueste Version von KOMA-Skript verwenden
]{scrlttr2}

% ===== Deutsche Sprache =====
\usepackage[utf8]{inputenc}
\usepackage[ngerman]{babel}
% ============================

\LoadLetterOption{DIN} % Einstellungen für DIN 676 laden

% \LoadLetterOption{absender} % Absenderdaten und -einstellungen aus absender.lco laden





\KOMAoptions{%
% fromemail=true,       % Email wird im Briefkopf angezeigt
% fromphone=true,       % Telefonnumer wird im Briefkopf angezeigt
% fromfax=true,         % Faxnummer wird im Briefkopf angezeit
% fromurl=true,         % URL wird im Briefkopf angezeigt
% fromlogo=true,        % Logo wird im Briefkopf angezeigt
% subject=titled,       % Druckt "Betrifft: " vor dem Betreff
locfield=wide,          % Breite Absenderergänzung (location)
fromalign=left,         % Ausrichtung des Briefkopfes
fromrule=afteraddress%  % Trennlinie unter dem Briefkopf
}

\RequirePackage[utf8]{inputenc}
\RequirePackage[ngerman]{babel}

\setkomavar{fromname}{$fromname$} % Name
\setkomavar{fromaddress}{% % Adresse
  {$for(fromaddress)$$fromaddress$$sep$\\$endfor$}
}
\setkomavar{fromfax}{$fromfax$} % Faxnummer
\setkomavar{fromemail}{$email$} % Email-Adresse
\setkomavar{fromphone}{$fromphone$} % Telefonnummer
\setkomavar{fromurl}[Website:~]{$fromurl$} % Website

% ===== Absenderergänzung =====
\setkomavar{location}{%
  \raggedright\footnotesize{%
  \usekomavar{fromname}\\
  \usekomavar{fromaddress}\\
  \usekomavar*{fromphone}\usekomavar{fromphone}\\
  \usekomavar*{fromfax}\usekomavar{fromfax}\\
  \usekomavar*{fromemail}\usekomavar{fromemail}
  \usekomavar*{fromurl}\usekomavar{fromurl}}%
}
% ============================

% Logo
% \setkomavar{fromlogo}{\includegraphics{logo.png}}

% Die Bankverbindung wird nicht automatisch verwendet. Dazu muss bspw. mittels \firstfoot ein eigener Brieffuß definiert werden.
\setkomavar{frombank}{}

% ===== Signatur =====
\setkomavar{signature}{%
  \usekomavar{fromname}\\
%
}
\renewcommand*{\raggedsignature}{\raggedright}
% ====================






\usepackage{graphicx} % Um Grafiken (bspw. das Logo) einbinden zu können

\begin{document}

\begin{letter}{%
% ===== Zielanschrift =====
{$for(address)$$address$$sep$\\$endfor$}
% =======================
}

% ====== Geschäftszeichenzeile =========
\setkomavar{yourref}{$yourref$}          % Ihr Zeichen
\setkomavar{yourmail}{}         % Ihr Schreiben vom
\setkomavar{myref}{}            % Unser Zeichen
\setkomavar{customer}{}         % Kundennummer
\setkomavar{invoice}{}          % Rechnungsnummer
\setkomavar{place}{$place$} % Ort
\setkomavar{date}{$date$}       % Datum
% =====================================

% \setkomavar{title}{Titel}

\setkomavar{subject}{$subject$}

\opening{$opening$}

$body$

\closing{$closing$}

% ===== Postskriptum =====
% \ps PS: \dots
% ========================

% ===== Anlage(n) =====
% \setkomavar*{enclseparator}{Anlage}
%\encl{%
%  Anlage 1\\
%  Anlage 2%
%}
% ===================

% ===== Verteiler =====
% \setkomavar*{ccseparator}{Kopie an}
%\cc{%
%  Verteiler 1\\
%  Verteiler 2%
%}
% =====================

\end{letter}
\end{document}
